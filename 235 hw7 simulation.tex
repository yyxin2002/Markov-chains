\documentclass[11pt]{article}

    \usepackage[breakable]{tcolorbox}
    \usepackage{parskip} % Stop auto-indenting (to mimic markdown behaviour)
    

    % Basic figure setup, for now with no caption control since it's done
    % automatically by Pandoc (which extracts ![](path) syntax from Markdown).
    \usepackage{graphicx}
    % Maintain compatibility with old templates. Remove in nbconvert 6.0
    \let\Oldincludegraphics\includegraphics
    % Ensure that by default, figures have no caption (until we provide a
    % proper Figure object with a Caption API and a way to capture that
    % in the conversion process - todo).
    \usepackage{caption}
    \DeclareCaptionFormat{nocaption}{}
    \captionsetup{format=nocaption,aboveskip=0pt,belowskip=0pt}

    \usepackage{float}
    \floatplacement{figure}{H} % forces figures to be placed at the correct location
    \usepackage{xcolor} % Allow colors to be defined
    \usepackage{enumerate} % Needed for markdown enumerations to work
    \usepackage{geometry} % Used to adjust the document margins
    \usepackage{amsmath} % Equations
    \usepackage{amssymb} % Equations
    \usepackage{textcomp} % defines textquotesingle
    % Hack from http://tex.stackexchange.com/a/47451/13684:
    \AtBeginDocument{%
        \def\PYZsq{\textquotesingle}% Upright quotes in Pygmentized code
    }
    \usepackage{upquote} % Upright quotes for verbatim code
    \usepackage{eurosym} % defines \euro

    \usepackage{iftex}
    \ifPDFTeX
        \usepackage[T1]{fontenc}
        \IfFileExists{alphabeta.sty}{
              \usepackage{alphabeta}
          }{
              \usepackage[mathletters]{ucs}
              \usepackage[utf8x]{inputenc}
          }
    \else
        \usepackage{fontspec}
        \usepackage{unicode-math}
    \fi

    \usepackage{fancyvrb} % verbatim replacement that allows latex
    \usepackage{grffile} % extends the file name processing of package graphics
                         % to support a larger range
    \makeatletter % fix for old versions of grffile with XeLaTeX
    \@ifpackagelater{grffile}{2019/11/01}
    {
      % Do nothing on new versions
    }
    {
      \def\Gread@@xetex#1{%
        \IfFileExists{"\Gin@base".bb}%
        {\Gread@eps{\Gin@base.bb}}%
        {\Gread@@xetex@aux#1}%
      }
    }
    \makeatother
    \usepackage[Export]{adjustbox} % Used to constrain images to a maximum size
    \adjustboxset{max size={0.9\linewidth}{0.9\paperheight}}

    % The hyperref package gives us a pdf with properly built
    % internal navigation ('pdf bookmarks' for the table of contents,
    % internal cross-reference links, web links for URLs, etc.)
    \usepackage{hyperref}
    % The default LaTeX title has an obnoxious amount of whitespace. By default,
    % titling removes some of it. It also provides customization options.
    \usepackage{titling}
    \usepackage{longtable} % longtable support required by pandoc >1.10
    \usepackage{booktabs}  % table support for pandoc > 1.12.2
    \usepackage{array}     % table support for pandoc >= 2.11.3
    \usepackage{calc}      % table minipage width calculation for pandoc >= 2.11.1
    \usepackage[inline]{enumitem} % IRkernel/repr support (it uses the enumerate* environment)
    \usepackage[normalem]{ulem} % ulem is needed to support strikethroughs (\sout)
                                % normalem makes italics be italics, not underlines
    \usepackage{mathrsfs}
    

    
    % Colors for the hyperref package
    \definecolor{urlcolor}{rgb}{0,.145,.698}
    \definecolor{linkcolor}{rgb}{.71,0.21,0.01}
    \definecolor{citecolor}{rgb}{.12,.54,.11}

    % ANSI colors
    \definecolor{ansi-black}{HTML}{3E424D}
    \definecolor{ansi-black-intense}{HTML}{282C36}
    \definecolor{ansi-red}{HTML}{E75C58}
    \definecolor{ansi-red-intense}{HTML}{B22B31}
    \definecolor{ansi-green}{HTML}{00A250}
    \definecolor{ansi-green-intense}{HTML}{007427}
    \definecolor{ansi-yellow}{HTML}{DDB62B}
    \definecolor{ansi-yellow-intense}{HTML}{B27D12}
    \definecolor{ansi-blue}{HTML}{208FFB}
    \definecolor{ansi-blue-intense}{HTML}{0065CA}
    \definecolor{ansi-magenta}{HTML}{D160C4}
    \definecolor{ansi-magenta-intense}{HTML}{A03196}
    \definecolor{ansi-cyan}{HTML}{60C6C8}
    \definecolor{ansi-cyan-intense}{HTML}{258F8F}
    \definecolor{ansi-white}{HTML}{C5C1B4}
    \definecolor{ansi-white-intense}{HTML}{A1A6B2}
    \definecolor{ansi-default-inverse-fg}{HTML}{FFFFFF}
    \definecolor{ansi-default-inverse-bg}{HTML}{000000}

    % common color for the border for error outputs.
    \definecolor{outerrorbackground}{HTML}{FFDFDF}

    % commands and environments needed by pandoc snippets
    % extracted from the output of `pandoc -s`
    \providecommand{\tightlist}{%
      \setlength{\itemsep}{0pt}\setlength{\parskip}{0pt}}
    \DefineVerbatimEnvironment{Highlighting}{Verbatim}{commandchars=\\\{\}}
    % Add ',fontsize=\small' for more characters per line
    \newenvironment{Shaded}{}{}
    \newcommand{\KeywordTok}[1]{\textcolor[rgb]{0.00,0.44,0.13}{\textbf{{#1}}}}
    \newcommand{\DataTypeTok}[1]{\textcolor[rgb]{0.56,0.13,0.00}{{#1}}}
    \newcommand{\DecValTok}[1]{\textcolor[rgb]{0.25,0.63,0.44}{{#1}}}
    \newcommand{\BaseNTok}[1]{\textcolor[rgb]{0.25,0.63,0.44}{{#1}}}
    \newcommand{\FloatTok}[1]{\textcolor[rgb]{0.25,0.63,0.44}{{#1}}}
    \newcommand{\CharTok}[1]{\textcolor[rgb]{0.25,0.44,0.63}{{#1}}}
    \newcommand{\StringTok}[1]{\textcolor[rgb]{0.25,0.44,0.63}{{#1}}}
    \newcommand{\CommentTok}[1]{\textcolor[rgb]{0.38,0.63,0.69}{\textit{{#1}}}}
    \newcommand{\OtherTok}[1]{\textcolor[rgb]{0.00,0.44,0.13}{{#1}}}
    \newcommand{\AlertTok}[1]{\textcolor[rgb]{1.00,0.00,0.00}{\textbf{{#1}}}}
    \newcommand{\FunctionTok}[1]{\textcolor[rgb]{0.02,0.16,0.49}{{#1}}}
    \newcommand{\RegionMarkerTok}[1]{{#1}}
    \newcommand{\ErrorTok}[1]{\textcolor[rgb]{1.00,0.00,0.00}{\textbf{{#1}}}}
    \newcommand{\NormalTok}[1]{{#1}}

    % Additional commands for more recent versions of Pandoc
    \newcommand{\ConstantTok}[1]{\textcolor[rgb]{0.53,0.00,0.00}{{#1}}}
    \newcommand{\SpecialCharTok}[1]{\textcolor[rgb]{0.25,0.44,0.63}{{#1}}}
    \newcommand{\VerbatimStringTok}[1]{\textcolor[rgb]{0.25,0.44,0.63}{{#1}}}
    \newcommand{\SpecialStringTok}[1]{\textcolor[rgb]{0.73,0.40,0.53}{{#1}}}
    \newcommand{\ImportTok}[1]{{#1}}
    \newcommand{\DocumentationTok}[1]{\textcolor[rgb]{0.73,0.13,0.13}{\textit{{#1}}}}
    \newcommand{\AnnotationTok}[1]{\textcolor[rgb]{0.38,0.63,0.69}{\textbf{\textit{{#1}}}}}
    \newcommand{\CommentVarTok}[1]{\textcolor[rgb]{0.38,0.63,0.69}{\textbf{\textit{{#1}}}}}
    \newcommand{\VariableTok}[1]{\textcolor[rgb]{0.10,0.09,0.49}{{#1}}}
    \newcommand{\ControlFlowTok}[1]{\textcolor[rgb]{0.00,0.44,0.13}{\textbf{{#1}}}}
    \newcommand{\OperatorTok}[1]{\textcolor[rgb]{0.40,0.40,0.40}{{#1}}}
    \newcommand{\BuiltInTok}[1]{{#1}}
    \newcommand{\ExtensionTok}[1]{{#1}}
    \newcommand{\PreprocessorTok}[1]{\textcolor[rgb]{0.74,0.48,0.00}{{#1}}}
    \newcommand{\AttributeTok}[1]{\textcolor[rgb]{0.49,0.56,0.16}{{#1}}}
    \newcommand{\InformationTok}[1]{\textcolor[rgb]{0.38,0.63,0.69}{\textbf{\textit{{#1}}}}}
    \newcommand{\WarningTok}[1]{\textcolor[rgb]{0.38,0.63,0.69}{\textbf{\textit{{#1}}}}}


    % Define a nice break command that doesn't care if a line doesn't already
    % exist.
    \def\br{\hspace*{\fill} \\* }
    % Math Jax compatibility definitions
    \def\gt{>}
    \def\lt{<}
    \let\Oldtex\TeX
    \let\Oldlatex\LaTeX
    \renewcommand{\TeX}{\textrm{\Oldtex}}
    \renewcommand{\LaTeX}{\textrm{\Oldlatex}}
    % Document parameters
    % Document title
    \title{235 hw7 simulation}
    
    
    
    
    
% Pygments definitions
\makeatletter
\def\PY@reset{\let\PY@it=\relax \let\PY@bf=\relax%
    \let\PY@ul=\relax \let\PY@tc=\relax%
    \let\PY@bc=\relax \let\PY@ff=\relax}
\def\PY@tok#1{\csname PY@tok@#1\endcsname}
\def\PY@toks#1+{\ifx\relax#1\empty\else%
    \PY@tok{#1}\expandafter\PY@toks\fi}
\def\PY@do#1{\PY@bc{\PY@tc{\PY@ul{%
    \PY@it{\PY@bf{\PY@ff{#1}}}}}}}
\def\PY#1#2{\PY@reset\PY@toks#1+\relax+\PY@do{#2}}

\@namedef{PY@tok@w}{\def\PY@tc##1{\textcolor[rgb]{0.73,0.73,0.73}{##1}}}
\@namedef{PY@tok@c}{\let\PY@it=\textit\def\PY@tc##1{\textcolor[rgb]{0.24,0.48,0.48}{##1}}}
\@namedef{PY@tok@cp}{\def\PY@tc##1{\textcolor[rgb]{0.61,0.40,0.00}{##1}}}
\@namedef{PY@tok@k}{\let\PY@bf=\textbf\def\PY@tc##1{\textcolor[rgb]{0.00,0.50,0.00}{##1}}}
\@namedef{PY@tok@kp}{\def\PY@tc##1{\textcolor[rgb]{0.00,0.50,0.00}{##1}}}
\@namedef{PY@tok@kt}{\def\PY@tc##1{\textcolor[rgb]{0.69,0.00,0.25}{##1}}}
\@namedef{PY@tok@o}{\def\PY@tc##1{\textcolor[rgb]{0.40,0.40,0.40}{##1}}}
\@namedef{PY@tok@ow}{\let\PY@bf=\textbf\def\PY@tc##1{\textcolor[rgb]{0.67,0.13,1.00}{##1}}}
\@namedef{PY@tok@nb}{\def\PY@tc##1{\textcolor[rgb]{0.00,0.50,0.00}{##1}}}
\@namedef{PY@tok@nf}{\def\PY@tc##1{\textcolor[rgb]{0.00,0.00,1.00}{##1}}}
\@namedef{PY@tok@nc}{\let\PY@bf=\textbf\def\PY@tc##1{\textcolor[rgb]{0.00,0.00,1.00}{##1}}}
\@namedef{PY@tok@nn}{\let\PY@bf=\textbf\def\PY@tc##1{\textcolor[rgb]{0.00,0.00,1.00}{##1}}}
\@namedef{PY@tok@ne}{\let\PY@bf=\textbf\def\PY@tc##1{\textcolor[rgb]{0.80,0.25,0.22}{##1}}}
\@namedef{PY@tok@nv}{\def\PY@tc##1{\textcolor[rgb]{0.10,0.09,0.49}{##1}}}
\@namedef{PY@tok@no}{\def\PY@tc##1{\textcolor[rgb]{0.53,0.00,0.00}{##1}}}
\@namedef{PY@tok@nl}{\def\PY@tc##1{\textcolor[rgb]{0.46,0.46,0.00}{##1}}}
\@namedef{PY@tok@ni}{\let\PY@bf=\textbf\def\PY@tc##1{\textcolor[rgb]{0.44,0.44,0.44}{##1}}}
\@namedef{PY@tok@na}{\def\PY@tc##1{\textcolor[rgb]{0.41,0.47,0.13}{##1}}}
\@namedef{PY@tok@nt}{\let\PY@bf=\textbf\def\PY@tc##1{\textcolor[rgb]{0.00,0.50,0.00}{##1}}}
\@namedef{PY@tok@nd}{\def\PY@tc##1{\textcolor[rgb]{0.67,0.13,1.00}{##1}}}
\@namedef{PY@tok@s}{\def\PY@tc##1{\textcolor[rgb]{0.73,0.13,0.13}{##1}}}
\@namedef{PY@tok@sd}{\let\PY@it=\textit\def\PY@tc##1{\textcolor[rgb]{0.73,0.13,0.13}{##1}}}
\@namedef{PY@tok@si}{\let\PY@bf=\textbf\def\PY@tc##1{\textcolor[rgb]{0.64,0.35,0.47}{##1}}}
\@namedef{PY@tok@se}{\let\PY@bf=\textbf\def\PY@tc##1{\textcolor[rgb]{0.67,0.36,0.12}{##1}}}
\@namedef{PY@tok@sr}{\def\PY@tc##1{\textcolor[rgb]{0.64,0.35,0.47}{##1}}}
\@namedef{PY@tok@ss}{\def\PY@tc##1{\textcolor[rgb]{0.10,0.09,0.49}{##1}}}
\@namedef{PY@tok@sx}{\def\PY@tc##1{\textcolor[rgb]{0.00,0.50,0.00}{##1}}}
\@namedef{PY@tok@m}{\def\PY@tc##1{\textcolor[rgb]{0.40,0.40,0.40}{##1}}}
\@namedef{PY@tok@gh}{\let\PY@bf=\textbf\def\PY@tc##1{\textcolor[rgb]{0.00,0.00,0.50}{##1}}}
\@namedef{PY@tok@gu}{\let\PY@bf=\textbf\def\PY@tc##1{\textcolor[rgb]{0.50,0.00,0.50}{##1}}}
\@namedef{PY@tok@gd}{\def\PY@tc##1{\textcolor[rgb]{0.63,0.00,0.00}{##1}}}
\@namedef{PY@tok@gi}{\def\PY@tc##1{\textcolor[rgb]{0.00,0.52,0.00}{##1}}}
\@namedef{PY@tok@gr}{\def\PY@tc##1{\textcolor[rgb]{0.89,0.00,0.00}{##1}}}
\@namedef{PY@tok@ge}{\let\PY@it=\textit}
\@namedef{PY@tok@gs}{\let\PY@bf=\textbf}
\@namedef{PY@tok@gp}{\let\PY@bf=\textbf\def\PY@tc##1{\textcolor[rgb]{0.00,0.00,0.50}{##1}}}
\@namedef{PY@tok@go}{\def\PY@tc##1{\textcolor[rgb]{0.44,0.44,0.44}{##1}}}
\@namedef{PY@tok@gt}{\def\PY@tc##1{\textcolor[rgb]{0.00,0.27,0.87}{##1}}}
\@namedef{PY@tok@err}{\def\PY@bc##1{{\setlength{\fboxsep}{\string -\fboxrule}\fcolorbox[rgb]{1.00,0.00,0.00}{1,1,1}{\strut ##1}}}}
\@namedef{PY@tok@kc}{\let\PY@bf=\textbf\def\PY@tc##1{\textcolor[rgb]{0.00,0.50,0.00}{##1}}}
\@namedef{PY@tok@kd}{\let\PY@bf=\textbf\def\PY@tc##1{\textcolor[rgb]{0.00,0.50,0.00}{##1}}}
\@namedef{PY@tok@kn}{\let\PY@bf=\textbf\def\PY@tc##1{\textcolor[rgb]{0.00,0.50,0.00}{##1}}}
\@namedef{PY@tok@kr}{\let\PY@bf=\textbf\def\PY@tc##1{\textcolor[rgb]{0.00,0.50,0.00}{##1}}}
\@namedef{PY@tok@bp}{\def\PY@tc##1{\textcolor[rgb]{0.00,0.50,0.00}{##1}}}
\@namedef{PY@tok@fm}{\def\PY@tc##1{\textcolor[rgb]{0.00,0.00,1.00}{##1}}}
\@namedef{PY@tok@vc}{\def\PY@tc##1{\textcolor[rgb]{0.10,0.09,0.49}{##1}}}
\@namedef{PY@tok@vg}{\def\PY@tc##1{\textcolor[rgb]{0.10,0.09,0.49}{##1}}}
\@namedef{PY@tok@vi}{\def\PY@tc##1{\textcolor[rgb]{0.10,0.09,0.49}{##1}}}
\@namedef{PY@tok@vm}{\def\PY@tc##1{\textcolor[rgb]{0.10,0.09,0.49}{##1}}}
\@namedef{PY@tok@sa}{\def\PY@tc##1{\textcolor[rgb]{0.73,0.13,0.13}{##1}}}
\@namedef{PY@tok@sb}{\def\PY@tc##1{\textcolor[rgb]{0.73,0.13,0.13}{##1}}}
\@namedef{PY@tok@sc}{\def\PY@tc##1{\textcolor[rgb]{0.73,0.13,0.13}{##1}}}
\@namedef{PY@tok@dl}{\def\PY@tc##1{\textcolor[rgb]{0.73,0.13,0.13}{##1}}}
\@namedef{PY@tok@s2}{\def\PY@tc##1{\textcolor[rgb]{0.73,0.13,0.13}{##1}}}
\@namedef{PY@tok@sh}{\def\PY@tc##1{\textcolor[rgb]{0.73,0.13,0.13}{##1}}}
\@namedef{PY@tok@s1}{\def\PY@tc##1{\textcolor[rgb]{0.73,0.13,0.13}{##1}}}
\@namedef{PY@tok@mb}{\def\PY@tc##1{\textcolor[rgb]{0.40,0.40,0.40}{##1}}}
\@namedef{PY@tok@mf}{\def\PY@tc##1{\textcolor[rgb]{0.40,0.40,0.40}{##1}}}
\@namedef{PY@tok@mh}{\def\PY@tc##1{\textcolor[rgb]{0.40,0.40,0.40}{##1}}}
\@namedef{PY@tok@mi}{\def\PY@tc##1{\textcolor[rgb]{0.40,0.40,0.40}{##1}}}
\@namedef{PY@tok@il}{\def\PY@tc##1{\textcolor[rgb]{0.40,0.40,0.40}{##1}}}
\@namedef{PY@tok@mo}{\def\PY@tc##1{\textcolor[rgb]{0.40,0.40,0.40}{##1}}}
\@namedef{PY@tok@ch}{\let\PY@it=\textit\def\PY@tc##1{\textcolor[rgb]{0.24,0.48,0.48}{##1}}}
\@namedef{PY@tok@cm}{\let\PY@it=\textit\def\PY@tc##1{\textcolor[rgb]{0.24,0.48,0.48}{##1}}}
\@namedef{PY@tok@cpf}{\let\PY@it=\textit\def\PY@tc##1{\textcolor[rgb]{0.24,0.48,0.48}{##1}}}
\@namedef{PY@tok@c1}{\let\PY@it=\textit\def\PY@tc##1{\textcolor[rgb]{0.24,0.48,0.48}{##1}}}
\@namedef{PY@tok@cs}{\let\PY@it=\textit\def\PY@tc##1{\textcolor[rgb]{0.24,0.48,0.48}{##1}}}

\def\PYZbs{\char`\\}
\def\PYZus{\char`\_}
\def\PYZob{\char`\{}
\def\PYZcb{\char`\}}
\def\PYZca{\char`\^}
\def\PYZam{\char`\&}
\def\PYZlt{\char`\<}
\def\PYZgt{\char`\>}
\def\PYZsh{\char`\#}
\def\PYZpc{\char`\%}
\def\PYZdl{\char`\$}
\def\PYZhy{\char`\-}
\def\PYZsq{\char`\'}
\def\PYZdq{\char`\"}
\def\PYZti{\char`\~}
% for compatibility with earlier versions
\def\PYZat{@}
\def\PYZlb{[}
\def\PYZrb{]}
\makeatother


    % For linebreaks inside Verbatim environment from package fancyvrb.
    \makeatletter
        \newbox\Wrappedcontinuationbox
        \newbox\Wrappedvisiblespacebox
        \newcommand*\Wrappedvisiblespace {\textcolor{red}{\textvisiblespace}}
        \newcommand*\Wrappedcontinuationsymbol {\textcolor{red}{\llap{\tiny$\m@th\hookrightarrow$}}}
        \newcommand*\Wrappedcontinuationindent {3ex }
        \newcommand*\Wrappedafterbreak {\kern\Wrappedcontinuationindent\copy\Wrappedcontinuationbox}
        % Take advantage of the already applied Pygments mark-up to insert
        % potential linebreaks for TeX processing.
        %        {, <, #, %, $, ' and ": go to next line.
        %        _, }, ^, &, >, - and ~: stay at end of broken line.
        % Use of \textquotesingle for straight quote.
        \newcommand*\Wrappedbreaksatspecials {%
            \def\PYGZus{\discretionary{\char`\_}{\Wrappedafterbreak}{\char`\_}}%
            \def\PYGZob{\discretionary{}{\Wrappedafterbreak\char`\{}{\char`\{}}%
            \def\PYGZcb{\discretionary{\char`\}}{\Wrappedafterbreak}{\char`\}}}%
            \def\PYGZca{\discretionary{\char`\^}{\Wrappedafterbreak}{\char`\^}}%
            \def\PYGZam{\discretionary{\char`\&}{\Wrappedafterbreak}{\char`\&}}%
            \def\PYGZlt{\discretionary{}{\Wrappedafterbreak\char`\<}{\char`\<}}%
            \def\PYGZgt{\discretionary{\char`\>}{\Wrappedafterbreak}{\char`\>}}%
            \def\PYGZsh{\discretionary{}{\Wrappedafterbreak\char`\#}{\char`\#}}%
            \def\PYGZpc{\discretionary{}{\Wrappedafterbreak\char`\%}{\char`\%}}%
            \def\PYGZdl{\discretionary{}{\Wrappedafterbreak\char`\$}{\char`\$}}%
            \def\PYGZhy{\discretionary{\char`\-}{\Wrappedafterbreak}{\char`\-}}%
            \def\PYGZsq{\discretionary{}{\Wrappedafterbreak\textquotesingle}{\textquotesingle}}%
            \def\PYGZdq{\discretionary{}{\Wrappedafterbreak\char`\"}{\char`\"}}%
            \def\PYGZti{\discretionary{\char`\~}{\Wrappedafterbreak}{\char`\~}}%
        }
        % Some characters . , ; ? ! / are not pygmentized.
        % This macro makes them "active" and they will insert potential linebreaks
        \newcommand*\Wrappedbreaksatpunct {%
            \lccode`\~`\.\lowercase{\def~}{\discretionary{\hbox{\char`\.}}{\Wrappedafterbreak}{\hbox{\char`\.}}}%
            \lccode`\~`\,\lowercase{\def~}{\discretionary{\hbox{\char`\,}}{\Wrappedafterbreak}{\hbox{\char`\,}}}%
            \lccode`\~`\;\lowercase{\def~}{\discretionary{\hbox{\char`\;}}{\Wrappedafterbreak}{\hbox{\char`\;}}}%
            \lccode`\~`\:\lowercase{\def~}{\discretionary{\hbox{\char`\:}}{\Wrappedafterbreak}{\hbox{\char`\:}}}%
            \lccode`\~`\?\lowercase{\def~}{\discretionary{\hbox{\char`\?}}{\Wrappedafterbreak}{\hbox{\char`\?}}}%
            \lccode`\~`\!\lowercase{\def~}{\discretionary{\hbox{\char`\!}}{\Wrappedafterbreak}{\hbox{\char`\!}}}%
            \lccode`\~`\/\lowercase{\def~}{\discretionary{\hbox{\char`\/}}{\Wrappedafterbreak}{\hbox{\char`\/}}}%
            \catcode`\.\active
            \catcode`\,\active
            \catcode`\;\active
            \catcode`\:\active
            \catcode`\?\active
            \catcode`\!\active
            \catcode`\/\active
            \lccode`\~`\~
        }
    \makeatother

    \let\OriginalVerbatim=\Verbatim
    \makeatletter
    \renewcommand{\Verbatim}[1][1]{%
        %\parskip\z@skip
        \sbox\Wrappedcontinuationbox {\Wrappedcontinuationsymbol}%
        \sbox\Wrappedvisiblespacebox {\FV@SetupFont\Wrappedvisiblespace}%
        \def\FancyVerbFormatLine ##1{\hsize\linewidth
            \vtop{\raggedright\hyphenpenalty\z@\exhyphenpenalty\z@
                \doublehyphendemerits\z@\finalhyphendemerits\z@
                \strut ##1\strut}%
        }%
        % If the linebreak is at a space, the latter will be displayed as visible
        % space at end of first line, and a continuation symbol starts next line.
        % Stretch/shrink are however usually zero for typewriter font.
        \def\FV@Space {%
            \nobreak\hskip\z@ plus\fontdimen3\font minus\fontdimen4\font
            \discretionary{\copy\Wrappedvisiblespacebox}{\Wrappedafterbreak}
            {\kern\fontdimen2\font}%
        }%

        % Allow breaks at special characters using \PYG... macros.
        \Wrappedbreaksatspecials
        % Breaks at punctuation characters . , ; ? ! and / need catcode=\active
        \OriginalVerbatim[#1,codes*=\Wrappedbreaksatpunct]%
    }
    \makeatother

    % Exact colors from NB
    \definecolor{incolor}{HTML}{303F9F}
    \definecolor{outcolor}{HTML}{D84315}
    \definecolor{cellborder}{HTML}{CFCFCF}
    \definecolor{cellbackground}{HTML}{F7F7F7}

    % prompt
    \makeatletter
    \newcommand{\boxspacing}{\kern\kvtcb@left@rule\kern\kvtcb@boxsep}
    \makeatother
    \newcommand{\prompt}[4]{
        {\ttfamily\llap{{\color{#2}[#3]:\hspace{3pt}#4}}\vspace{-\baselineskip}}
    }
    

    
    % Prevent overflowing lines due to hard-to-break entities
    \sloppy
    % Setup hyperref package
    \hypersetup{
      breaklinks=true,  % so long urls are correctly broken across lines
      colorlinks=true,
      urlcolor=urlcolor,
      linkcolor=linkcolor,
      citecolor=citecolor,
      }
    % Slightly bigger margins than the latex defaults
    
    \geometry{verbose,tmargin=1in,bmargin=1in,lmargin=1in,rmargin=1in}
    
    

\begin{document}
    
    \maketitle
    
    

    
    \begin{tcolorbox}[breakable, size=fbox, boxrule=1pt, pad at break*=1mm,colback=cellbackground, colframe=cellborder]
\prompt{In}{incolor}{14}{\boxspacing}
\begin{Verbatim}[commandchars=\\\{\}]
\PY{k+kn}{import} \PY{n+nn}{random}

\PY{c+c1}{\PYZsh{} initialize the urn with one red and one green ball}
\PY{n}{red\PYZus{}balls} \PY{o}{=} \PY{l+m+mi}{1}
\PY{n}{green\PYZus{}balls} \PY{o}{=} \PY{l+m+mi}{1}

\PY{c+c1}{\PYZsh{} continue until the number of balls in the urn is 1000}
\PY{k}{while} \PY{n}{red\PYZus{}balls} \PY{o}{+} \PY{n}{green\PYZus{}balls} \PY{o}{\PYZlt{}} \PY{l+m+mi}{1000}\PY{p}{:}
    \PY{c+c1}{\PYZsh{} calculate the probabilities of drawing a red or green ball}
    \PY{n}{total\PYZus{}balls} \PY{o}{=} \PY{n}{red\PYZus{}balls} \PY{o}{+} \PY{n}{green\PYZus{}balls}
    \PY{n}{red\PYZus{}prob} \PY{o}{=} \PY{n}{red\PYZus{}balls} \PY{o}{/} \PY{n}{total\PYZus{}balls}
    \PY{n}{green\PYZus{}prob} \PY{o}{=} \PY{n}{green\PYZus{}balls} \PY{o}{/} \PY{n}{total\PYZus{}balls}
    
    \PY{c+c1}{\PYZsh{} randomly draw a ball from the urn}
    \PY{k}{if} \PY{n}{random}\PY{o}{.}\PY{n}{random}\PY{p}{(}\PY{p}{)} \PY{o}{\PYZlt{}} \PY{n}{red\PYZus{}prob}\PY{p}{:}
        \PY{c+c1}{\PYZsh{} the drawn ball is red}
        \PY{n}{red\PYZus{}balls} \PY{o}{+}\PY{o}{=} \PY{l+m+mi}{1}
    \PY{k}{else}\PY{p}{:}
        \PY{c+c1}{\PYZsh{} the drawn ball is green}
        \PY{n}{green\PYZus{}balls} \PY{o}{+}\PY{o}{=} \PY{l+m+mi}{1}

\PY{c+c1}{\PYZsh{} calculate the fraction of red balls in the urn}
\PY{n}{red\PYZus{}fraction} \PY{o}{=} \PY{n}{red\PYZus{}balls} \PY{o}{/} \PY{p}{(}\PY{n}{red\PYZus{}balls} \PY{o}{+} \PY{n}{green\PYZus{}balls}\PY{p}{)}
\PY{n+nb}{print}\PY{p}{(}\PY{l+s+sa}{f}\PY{l+s+s2}{\PYZdq{}}\PY{l+s+s2}{Fraction of red balls: }\PY{l+s+si}{\PYZob{}}\PY{n}{red\PYZus{}fraction}\PY{l+s+si}{\PYZcb{}}\PY{l+s+s2}{\PYZdq{}}\PY{p}{)}
\end{Verbatim}
\end{tcolorbox}

    \begin{Verbatim}[commandchars=\\\{\}]
Fraction of red balls: 0.199
    \end{Verbatim}

    \begin{tcolorbox}[breakable, size=fbox, boxrule=1pt, pad at break*=1mm,colback=cellbackground, colframe=cellborder]
\prompt{In}{incolor}{18}{\boxspacing}
\begin{Verbatim}[commandchars=\\\{\}]
\PY{k+kn}{import} \PY{n+nn}{random}
\PY{k+kn}{import} \PY{n+nn}{numpy} \PY{k}{as} \PY{n+nn}{np}

\PY{c+c1}{\PYZsh{} define the intervals for red ball fractions}
\PY{n}{intervals} \PY{o}{=} \PY{n}{np}\PY{o}{.}\PY{n}{linspace}\PY{p}{(}\PY{l+m+mi}{0}\PY{p}{,} \PY{l+m+mi}{1}\PY{p}{,} \PY{l+m+mi}{21}\PY{p}{)}

\PY{c+c1}{\PYZsh{} initialize a list to store the fraction of red balls in each simulation}
\PY{n}{red\PYZus{}fractions} \PY{o}{=} \PY{p}{[}\PY{p}{]}

\PY{c+c1}{\PYZsh{} repeat the simulation 2000 times}
\PY{k}{for} \PY{n}{i} \PY{o+ow}{in} \PY{n+nb}{range}\PY{p}{(}\PY{l+m+mi}{2000}\PY{p}{)}\PY{p}{:}
    \PY{c+c1}{\PYZsh{} initialize the urn with one red and one green ball}
    \PY{n}{red\PYZus{}balls} \PY{o}{=} \PY{l+m+mi}{1}
    \PY{n}{green\PYZus{}balls} \PY{o}{=} \PY{l+m+mi}{1}
    
    \PY{c+c1}{\PYZsh{} continue until the number of balls in the urn is 2000}
    \PY{k}{while} \PY{n}{red\PYZus{}balls} \PY{o}{+} \PY{n}{green\PYZus{}balls} \PY{o}{\PYZlt{}} \PY{l+m+mi}{2000}\PY{p}{:}
        \PY{c+c1}{\PYZsh{} calculate the probabilities of drawing a red or green ball}
        \PY{n}{total\PYZus{}balls} \PY{o}{=} \PY{n}{red\PYZus{}balls} \PY{o}{+} \PY{n}{green\PYZus{}balls}
        \PY{n}{red\PYZus{}prob} \PY{o}{=} \PY{n}{red\PYZus{}balls} \PY{o}{/} \PY{n}{total\PYZus{}balls}
        \PY{n}{green\PYZus{}prob} \PY{o}{=} \PY{n}{green\PYZus{}balls} \PY{o}{/} \PY{n}{total\PYZus{}balls}
        
        \PY{c+c1}{\PYZsh{} randomly draw a ball from the urn}
        \PY{k}{if} \PY{n}{random}\PY{o}{.}\PY{n}{random}\PY{p}{(}\PY{p}{)} \PY{o}{\PYZlt{}} \PY{n}{red\PYZus{}prob}\PY{p}{:}
            \PY{c+c1}{\PYZsh{} the drawn ball is red}
            \PY{n}{red\PYZus{}balls} \PY{o}{+}\PY{o}{=} \PY{l+m+mi}{1}
        \PY{k}{else}\PY{p}{:}
            \PY{c+c1}{\PYZsh{} the drawn ball is green}
            \PY{n}{green\PYZus{}balls} \PY{o}{+}\PY{o}{=} \PY{l+m+mi}{1}
    
    \PY{c+c1}{\PYZsh{} calculate the fraction of red balls in the urn}
    \PY{n}{red\PYZus{}fraction} \PY{o}{=} \PY{n}{red\PYZus{}balls} \PY{o}{/} \PY{p}{(}\PY{n}{red\PYZus{}balls} \PY{o}{+} \PY{n}{green\PYZus{}balls}\PY{p}{)}
    \PY{n}{red\PYZus{}fractions}\PY{o}{.}\PY{n}{append}\PY{p}{(}\PY{n}{red\PYZus{}fraction}\PY{p}{)}

\PY{c+c1}{\PYZsh{} compute the frequency of red ball fractions in each interval}
\PY{n}{freq}\PY{p}{,} \PY{n}{\PYZus{}} \PY{o}{=} \PY{n}{np}\PY{o}{.}\PY{n}{histogram}\PY{p}{(}\PY{n}{red\PYZus{}fractions}\PY{p}{,} \PY{n}{bins}\PY{o}{=}\PY{n}{intervals}\PY{p}{)}

\PY{c+c1}{\PYZsh{} print the number of times the fraction of red balls falls within each interval}
\PY{k}{for} \PY{n}{i} \PY{o+ow}{in} \PY{n+nb}{range}\PY{p}{(}\PY{n+nb}{len}\PY{p}{(}\PY{n}{intervals}\PY{p}{)} \PY{o}{\PYZhy{}} \PY{l+m+mi}{1}\PY{p}{)}\PY{p}{:}
    \PY{n}{interval\PYZus{}str} \PY{o}{=} \PY{l+s+sa}{f}\PY{l+s+s2}{\PYZdq{}}\PY{l+s+s2}{[}\PY{l+s+si}{\PYZob{}}\PY{n}{intervals}\PY{p}{[}\PY{n}{i}\PY{p}{]}\PY{l+s+si}{:}\PY{l+s+s2}{.2f}\PY{l+s+si}{\PYZcb{}}\PY{l+s+s2}{, }\PY{l+s+si}{\PYZob{}}\PY{n}{intervals}\PY{p}{[}\PY{n}{i}\PY{o}{+}\PY{l+m+mi}{1}\PY{p}{]}\PY{l+s+si}{:}\PY{l+s+s2}{.2f}\PY{l+s+si}{\PYZcb{}}\PY{l+s+s2}{]}\PY{l+s+s2}{\PYZdq{}}
    \PY{n+nb}{print}\PY{p}{(}\PY{l+s+sa}{f}\PY{l+s+s2}{\PYZdq{}}\PY{l+s+si}{\PYZob{}}\PY{n}{interval\PYZus{}str}\PY{l+s+si}{\PYZcb{}}\PY{l+s+s2}{: }\PY{l+s+si}{\PYZob{}}\PY{n}{freq}\PY{p}{[}\PY{n}{i}\PY{p}{]}\PY{l+s+si}{\PYZcb{}}\PY{l+s+s2}{\PYZdq{}}\PY{p}{)}
\end{Verbatim}
\end{tcolorbox}

    \begin{Verbatim}[commandchars=\\\{\}]
[0.00, 0.05]: 113
[0.05, 0.10]: 97
[0.10, 0.15]: 116
[0.15, 0.20]: 90
[0.20, 0.25]: 76
[0.25, 0.30]: 107
[0.30, 0.35]: 109
[0.35, 0.40]: 97
[0.40, 0.45]: 107
[0.45, 0.50]: 92
[0.50, 0.55]: 109
[0.55, 0.60]: 115
[0.60, 0.65]: 89
[0.65, 0.70]: 110
[0.70, 0.75]: 101
[0.75, 0.80]: 112
[0.80, 0.85]: 100
[0.85, 0.90]: 80
[0.90, 0.95]: 90
[0.95, 1.00]: 90
    \end{Verbatim}

    \begin{tcolorbox}[breakable, size=fbox, boxrule=1pt, pad at break*=1mm,colback=cellbackground, colframe=cellborder]
\prompt{In}{incolor}{19}{\boxspacing}
\begin{Verbatim}[commandchars=\\\{\}]
\PY{k+kn}{import} \PY{n+nn}{random}

\PY{c+c1}{\PYZsh{} repeat the simulation 100 times}
\PY{k}{for} \PY{n}{i} \PY{o+ow}{in} \PY{n+nb}{range}\PY{p}{(}\PY{l+m+mi}{100}\PY{p}{)}\PY{p}{:}
    \PY{c+c1}{\PYZsh{} initialize the urn with one red and one green ball}
    \PY{n}{red\PYZus{}balls} \PY{o}{=} \PY{l+m+mi}{1}
    \PY{n}{green\PYZus{}balls} \PY{o}{=} \PY{l+m+mi}{1}
    
    \PY{c+c1}{\PYZsh{} continue until the number of balls in the urn is 1000}
    \PY{k}{while} \PY{n}{red\PYZus{}balls} \PY{o}{+} \PY{n}{green\PYZus{}balls} \PY{o}{\PYZlt{}} \PY{l+m+mi}{1000}\PY{p}{:}
        \PY{c+c1}{\PYZsh{} calculate the probabilities of drawing a red or green ball}
        \PY{n}{total\PYZus{}balls} \PY{o}{=} \PY{n}{red\PYZus{}balls} \PY{o}{+} \PY{n}{green\PYZus{}balls}
        \PY{n}{red\PYZus{}prob} \PY{o}{=} \PY{n}{red\PYZus{}balls} \PY{o}{/} \PY{n}{total\PYZus{}balls}
        \PY{n}{green\PYZus{}prob} \PY{o}{=} \PY{n}{green\PYZus{}balls} \PY{o}{/} \PY{n}{total\PYZus{}balls}
        
        \PY{c+c1}{\PYZsh{} randomly draw a ball from the urn}
        \PY{k}{if} \PY{n}{random}\PY{o}{.}\PY{n}{random}\PY{p}{(}\PY{p}{)} \PY{o}{\PYZlt{}} \PY{n}{red\PYZus{}prob}\PY{p}{:}
            \PY{c+c1}{\PYZsh{} the drawn ball is red}
            \PY{n}{red\PYZus{}balls} \PY{o}{+}\PY{o}{=} \PY{l+m+mi}{1}
        \PY{k}{else}\PY{p}{:}
            \PY{c+c1}{\PYZsh{} the drawn ball is green}
            \PY{n}{green\PYZus{}balls} \PY{o}{+}\PY{o}{=} \PY{l+m+mi}{1}
    
    \PY{c+c1}{\PYZsh{} calculate the proportion of red balls at 1000 balls}
    \PY{n}{red\PYZus{}prop\PYZus{}1000} \PY{o}{=} \PY{n}{red\PYZus{}balls} \PY{o}{/} \PY{p}{(}\PY{n}{red\PYZus{}balls} \PY{o}{+} \PY{n}{green\PYZus{}balls}\PY{p}{)}
    
    \PY{c+c1}{\PYZsh{} continue until the number of balls in the urn is 2000}
    \PY{k}{while} \PY{n}{red\PYZus{}balls} \PY{o}{+} \PY{n}{green\PYZus{}balls} \PY{o}{\PYZlt{}} \PY{l+m+mi}{2000}\PY{p}{:}
        \PY{c+c1}{\PYZsh{} calculate the probabilities of drawing a red or green ball}
        \PY{n}{total\PYZus{}balls} \PY{o}{=} \PY{n}{red\PYZus{}balls} \PY{o}{+} \PY{n}{green\PYZus{}balls}
        \PY{n}{red\PYZus{}prob} \PY{o}{=} \PY{n}{red\PYZus{}balls} \PY{o}{/} \PY{n}{total\PYZus{}balls}
        \PY{n}{green\PYZus{}prob} \PY{o}{=} \PY{n}{green\PYZus{}balls} \PY{o}{/} \PY{n}{total\PYZus{}balls}
        
        \PY{c+c1}{\PYZsh{} randomly draw a ball from the urn}
        \PY{k}{if} \PY{n}{random}\PY{o}{.}\PY{n}{random}\PY{p}{(}\PY{p}{)} \PY{o}{\PYZlt{}} \PY{n}{red\PYZus{}prob}\PY{p}{:}
            \PY{c+c1}{\PYZsh{} the drawn ball is red}
            \PY{n}{red\PYZus{}balls} \PY{o}{+}\PY{o}{=} \PY{l+m+mi}{1}
        \PY{k}{else}\PY{p}{:}
            \PY{c+c1}{\PYZsh{} the drawn ball is green}
            \PY{n}{green\PYZus{}balls} \PY{o}{+}\PY{o}{=} \PY{l+m+mi}{1}
    
    \PY{c+c1}{\PYZsh{} calculate the proportion of red balls at 2000 balls}
    \PY{n}{red\PYZus{}prop\PYZus{}2000} \PY{o}{=} \PY{n}{red\PYZus{}balls} \PY{o}{/} \PY{p}{(}\PY{n}{red\PYZus{}balls} \PY{o}{+} \PY{n}{green\PYZus{}balls}\PY{p}{)}
    
    \PY{c+c1}{\PYZsh{} print the comparison of proportions}
    \PY{n+nb}{print}\PY{p}{(}\PY{l+s+sa}{f}\PY{l+s+s2}{\PYZdq{}}\PY{l+s+s2}{Simulation }\PY{l+s+si}{\PYZob{}}\PY{n}{i}\PY{o}{+}\PY{l+m+mi}{1}\PY{l+s+si}{\PYZcb{}}\PY{l+s+s2}{: Proportion of red balls at 1000 balls: }\PY{l+s+si}{\PYZob{}}\PY{n}{red\PYZus{}prop\PYZus{}1000}\PY{l+s+si}{:}\PY{l+s+s2}{.4f}\PY{l+s+si}{\PYZcb{}}\PY{l+s+s2}{, Proportion of red balls at 2000 balls: }\PY{l+s+si}{\PYZob{}}\PY{n}{red\PYZus{}prop\PYZus{}2000}\PY{l+s+si}{:}\PY{l+s+s2}{.4f}\PY{l+s+si}{\PYZcb{}}\PY{l+s+s2}{, Difference: }\PY{l+s+si}{\PYZob{}}\PY{n+nb}{abs}\PY{p}{(}\PY{n}{red\PYZus{}prop\PYZus{}1000}\PY{+w}{ }\PY{o}{\PYZhy{}}\PY{+w}{ }\PY{n}{red\PYZus{}prop\PYZus{}2000}\PY{p}{)}\PY{l+s+si}{:}\PY{l+s+s2}{.4f}\PY{l+s+si}{\PYZcb{}}\PY{l+s+s2}{\PYZdq{}}\PY{p}{)}
\end{Verbatim}
\end{tcolorbox}

    \begin{Verbatim}[commandchars=\\\{\}]
Simulation 1: Proportion of red balls at 1000 balls: 0.8940, Proportion of red
balls at 2000 balls: 0.8945, Difference: 0.0005
Simulation 2: Proportion of red balls at 1000 balls: 0.5880, Proportion of red
balls at 2000 balls: 0.5835, Difference: 0.0045
Simulation 3: Proportion of red balls at 1000 balls: 0.1830, Proportion of red
balls at 2000 balls: 0.1860, Difference: 0.0030
Simulation 4: Proportion of red balls at 1000 balls: 0.4420, Proportion of red
balls at 2000 balls: 0.4555, Difference: 0.0135
Simulation 5: Proportion of red balls at 1000 balls: 0.3310, Proportion of red
balls at 2000 balls: 0.3140, Difference: 0.0170
Simulation 6: Proportion of red balls at 1000 balls: 0.3170, Proportion of red
balls at 2000 balls: 0.3020, Difference: 0.0150
Simulation 7: Proportion of red balls at 1000 balls: 0.8390, Proportion of red
balls at 2000 balls: 0.8415, Difference: 0.0025
Simulation 8: Proportion of red balls at 1000 balls: 0.7150, Proportion of red
balls at 2000 balls: 0.7160, Difference: 0.0010
Simulation 9: Proportion of red balls at 1000 balls: 0.0310, Proportion of red
balls at 2000 balls: 0.0360, Difference: 0.0050
Simulation 10: Proportion of red balls at 1000 balls: 0.3860, Proportion of red
balls at 2000 balls: 0.3750, Difference: 0.0110
Simulation 11: Proportion of red balls at 1000 balls: 0.9340, Proportion of red
balls at 2000 balls: 0.9180, Difference: 0.0160
Simulation 12: Proportion of red balls at 1000 balls: 0.8940, Proportion of red
balls at 2000 balls: 0.8990, Difference: 0.0050
Simulation 13: Proportion of red balls at 1000 balls: 0.3600, Proportion of red
balls at 2000 balls: 0.3515, Difference: 0.0085
Simulation 14: Proportion of red balls at 1000 balls: 0.1440, Proportion of red
balls at 2000 balls: 0.1325, Difference: 0.0115
Simulation 15: Proportion of red balls at 1000 balls: 0.5560, Proportion of red
balls at 2000 balls: 0.5700, Difference: 0.0140
Simulation 16: Proportion of red balls at 1000 balls: 0.6460, Proportion of red
balls at 2000 balls: 0.6360, Difference: 0.0100
Simulation 17: Proportion of red balls at 1000 balls: 0.4550, Proportion of red
balls at 2000 balls: 0.4700, Difference: 0.0150
Simulation 18: Proportion of red balls at 1000 balls: 0.5820, Proportion of red
balls at 2000 balls: 0.5745, Difference: 0.0075
Simulation 19: Proportion of red balls at 1000 balls: 0.1340, Proportion of red
balls at 2000 balls: 0.1335, Difference: 0.0005
Simulation 20: Proportion of red balls at 1000 balls: 0.5370, Proportion of red
balls at 2000 balls: 0.5430, Difference: 0.0060
Simulation 21: Proportion of red balls at 1000 balls: 0.5950, Proportion of red
balls at 2000 balls: 0.5840, Difference: 0.0110
Simulation 22: Proportion of red balls at 1000 balls: 0.1510, Proportion of red
balls at 2000 balls: 0.1470, Difference: 0.0040
Simulation 23: Proportion of red balls at 1000 balls: 0.4390, Proportion of red
balls at 2000 balls: 0.4415, Difference: 0.0025
Simulation 24: Proportion of red balls at 1000 balls: 0.9120, Proportion of red
balls at 2000 balls: 0.9240, Difference: 0.0120
Simulation 25: Proportion of red balls at 1000 balls: 0.8840, Proportion of red
balls at 2000 balls: 0.8905, Difference: 0.0065
Simulation 26: Proportion of red balls at 1000 balls: 0.6890, Proportion of red
balls at 2000 balls: 0.6835, Difference: 0.0055
Simulation 27: Proportion of red balls at 1000 balls: 0.7090, Proportion of red
balls at 2000 balls: 0.7135, Difference: 0.0045
Simulation 28: Proportion of red balls at 1000 balls: 0.8450, Proportion of red
balls at 2000 balls: 0.8280, Difference: 0.0170
Simulation 29: Proportion of red balls at 1000 balls: 0.1940, Proportion of red
balls at 2000 balls: 0.2005, Difference: 0.0065
Simulation 30: Proportion of red balls at 1000 balls: 0.0810, Proportion of red
balls at 2000 balls: 0.0805, Difference: 0.0005
Simulation 31: Proportion of red balls at 1000 balls: 0.8930, Proportion of red
balls at 2000 balls: 0.8750, Difference: 0.0180
Simulation 32: Proportion of red balls at 1000 balls: 0.2130, Proportion of red
balls at 2000 balls: 0.2035, Difference: 0.0095
Simulation 33: Proportion of red balls at 1000 balls: 0.3120, Proportion of red
balls at 2000 balls: 0.3260, Difference: 0.0140
Simulation 34: Proportion of red balls at 1000 balls: 0.6910, Proportion of red
balls at 2000 balls: 0.6905, Difference: 0.0005
Simulation 35: Proportion of red balls at 1000 balls: 0.4600, Proportion of red
balls at 2000 balls: 0.4645, Difference: 0.0045
Simulation 36: Proportion of red balls at 1000 balls: 0.9590, Proportion of red
balls at 2000 balls: 0.9600, Difference: 0.0010
Simulation 37: Proportion of red balls at 1000 balls: 0.5790, Proportion of red
balls at 2000 balls: 0.5575, Difference: 0.0215
Simulation 38: Proportion of red balls at 1000 balls: 0.7590, Proportion of red
balls at 2000 balls: 0.7705, Difference: 0.0115
Simulation 39: Proportion of red balls at 1000 balls: 0.3150, Proportion of red
balls at 2000 balls: 0.3130, Difference: 0.0020
Simulation 40: Proportion of red balls at 1000 balls: 0.9310, Proportion of red
balls at 2000 balls: 0.9340, Difference: 0.0030
Simulation 41: Proportion of red balls at 1000 balls: 0.4160, Proportion of red
balls at 2000 balls: 0.4155, Difference: 0.0005
Simulation 42: Proportion of red balls at 1000 balls: 0.2540, Proportion of red
balls at 2000 balls: 0.2460, Difference: 0.0080
Simulation 43: Proportion of red balls at 1000 balls: 0.7160, Proportion of red
balls at 2000 balls: 0.7270, Difference: 0.0110
Simulation 44: Proportion of red balls at 1000 balls: 0.8190, Proportion of red
balls at 2000 balls: 0.8190, Difference: 0.0000
Simulation 45: Proportion of red balls at 1000 balls: 0.5550, Proportion of red
balls at 2000 balls: 0.5565, Difference: 0.0015
Simulation 46: Proportion of red balls at 1000 balls: 0.7940, Proportion of red
balls at 2000 balls: 0.7800, Difference: 0.0140
Simulation 47: Proportion of red balls at 1000 balls: 0.7390, Proportion of red
balls at 2000 balls: 0.7545, Difference: 0.0155
Simulation 48: Proportion of red balls at 1000 balls: 0.5680, Proportion of red
balls at 2000 balls: 0.5665, Difference: 0.0015
Simulation 49: Proportion of red balls at 1000 balls: 0.5750, Proportion of red
balls at 2000 balls: 0.5810, Difference: 0.0060
Simulation 50: Proportion of red balls at 1000 balls: 0.6020, Proportion of red
balls at 2000 balls: 0.6165, Difference: 0.0145
Simulation 51: Proportion of red balls at 1000 balls: 0.0790, Proportion of red
balls at 2000 balls: 0.0915, Difference: 0.0125
Simulation 52: Proportion of red balls at 1000 balls: 0.4750, Proportion of red
balls at 2000 balls: 0.4870, Difference: 0.0120
Simulation 53: Proportion of red balls at 1000 balls: 0.3340, Proportion of red
balls at 2000 balls: 0.3175, Difference: 0.0165
Simulation 54: Proportion of red balls at 1000 balls: 0.1310, Proportion of red
balls at 2000 balls: 0.1265, Difference: 0.0045
Simulation 55: Proportion of red balls at 1000 balls: 0.1550, Proportion of red
balls at 2000 balls: 0.1620, Difference: 0.0070
Simulation 56: Proportion of red balls at 1000 balls: 0.0950, Proportion of red
balls at 2000 balls: 0.0950, Difference: 0.0000
Simulation 57: Proportion of red balls at 1000 balls: 0.6340, Proportion of red
balls at 2000 balls: 0.6105, Difference: 0.0235
Simulation 58: Proportion of red balls at 1000 balls: 0.9200, Proportion of red
balls at 2000 balls: 0.9160, Difference: 0.0040
Simulation 59: Proportion of red balls at 1000 balls: 0.7690, Proportion of red
balls at 2000 balls: 0.7825, Difference: 0.0135
Simulation 60: Proportion of red balls at 1000 balls: 0.2610, Proportion of red
balls at 2000 balls: 0.2605, Difference: 0.0005
Simulation 61: Proportion of red balls at 1000 balls: 0.5340, Proportion of red
balls at 2000 balls: 0.5470, Difference: 0.0130
Simulation 62: Proportion of red balls at 1000 balls: 0.2890, Proportion of red
balls at 2000 balls: 0.2815, Difference: 0.0075
Simulation 63: Proportion of red balls at 1000 balls: 0.7380, Proportion of red
balls at 2000 balls: 0.7340, Difference: 0.0040
Simulation 64: Proportion of red balls at 1000 balls: 0.2800, Proportion of red
balls at 2000 balls: 0.2750, Difference: 0.0050
Simulation 65: Proportion of red balls at 1000 balls: 0.8150, Proportion of red
balls at 2000 balls: 0.8115, Difference: 0.0035
Simulation 66: Proportion of red balls at 1000 balls: 0.2130, Proportion of red
balls at 2000 balls: 0.2035, Difference: 0.0095
Simulation 67: Proportion of red balls at 1000 balls: 0.1650, Proportion of red
balls at 2000 balls: 0.1705, Difference: 0.0055
Simulation 68: Proportion of red balls at 1000 balls: 0.5370, Proportion of red
balls at 2000 balls: 0.5370, Difference: 0.0000
Simulation 69: Proportion of red balls at 1000 balls: 0.5630, Proportion of red
balls at 2000 balls: 0.5525, Difference: 0.0105
Simulation 70: Proportion of red balls at 1000 balls: 0.5970, Proportion of red
balls at 2000 balls: 0.5875, Difference: 0.0095
Simulation 71: Proportion of red balls at 1000 balls: 0.8960, Proportion of red
balls at 2000 balls: 0.8970, Difference: 0.0010
Simulation 72: Proportion of red balls at 1000 balls: 0.1810, Proportion of red
balls at 2000 balls: 0.1705, Difference: 0.0105
Simulation 73: Proportion of red balls at 1000 balls: 0.0210, Proportion of red
balls at 2000 balls: 0.0185, Difference: 0.0025
Simulation 74: Proportion of red balls at 1000 balls: 0.7140, Proportion of red
balls at 2000 balls: 0.7150, Difference: 0.0010
Simulation 75: Proportion of red balls at 1000 balls: 0.5170, Proportion of red
balls at 2000 balls: 0.5155, Difference: 0.0015
Simulation 76: Proportion of red balls at 1000 balls: 0.1030, Proportion of red
balls at 2000 balls: 0.0980, Difference: 0.0050
Simulation 77: Proportion of red balls at 1000 balls: 0.0120, Proportion of red
balls at 2000 balls: 0.0135, Difference: 0.0015
Simulation 78: Proportion of red balls at 1000 balls: 0.0180, Proportion of red
balls at 2000 balls: 0.0165, Difference: 0.0015
Simulation 79: Proportion of red balls at 1000 balls: 0.0930, Proportion of red
balls at 2000 balls: 0.0915, Difference: 0.0015
Simulation 80: Proportion of red balls at 1000 balls: 0.4900, Proportion of red
balls at 2000 balls: 0.4675, Difference: 0.0225
Simulation 81: Proportion of red balls at 1000 balls: 0.1900, Proportion of red
balls at 2000 balls: 0.1885, Difference: 0.0015
Simulation 82: Proportion of red balls at 1000 balls: 0.7830, Proportion of red
balls at 2000 balls: 0.7945, Difference: 0.0115
Simulation 83: Proportion of red balls at 1000 balls: 0.7050, Proportion of red
balls at 2000 balls: 0.6900, Difference: 0.0150
Simulation 84: Proportion of red balls at 1000 balls: 0.4130, Proportion of red
balls at 2000 balls: 0.4245, Difference: 0.0115
Simulation 85: Proportion of red balls at 1000 balls: 0.9040, Proportion of red
balls at 2000 balls: 0.9040, Difference: 0.0000
Simulation 86: Proportion of red balls at 1000 balls: 0.2870, Proportion of red
balls at 2000 balls: 0.2870, Difference: 0.0000
Simulation 87: Proportion of red balls at 1000 balls: 0.5770, Proportion of red
balls at 2000 balls: 0.5510, Difference: 0.0260
Simulation 88: Proportion of red balls at 1000 balls: 0.7840, Proportion of red
balls at 2000 balls: 0.7680, Difference: 0.0160
Simulation 89: Proportion of red balls at 1000 balls: 0.8180, Proportion of red
balls at 2000 balls: 0.8075, Difference: 0.0105
Simulation 90: Proportion of red balls at 1000 balls: 0.1000, Proportion of red
balls at 2000 balls: 0.0985, Difference: 0.0015
Simulation 91: Proportion of red balls at 1000 balls: 0.9150, Proportion of red
balls at 2000 balls: 0.9095, Difference: 0.0055
Simulation 92: Proportion of red balls at 1000 balls: 0.4780, Proportion of red
balls at 2000 balls: 0.4735, Difference: 0.0045
Simulation 93: Proportion of red balls at 1000 balls: 0.7530, Proportion of red
balls at 2000 balls: 0.7365, Difference: 0.0165
Simulation 94: Proportion of red balls at 1000 balls: 0.9440, Proportion of red
balls at 2000 balls: 0.9500, Difference: 0.0060
Simulation 95: Proportion of red balls at 1000 balls: 0.3500, Proportion of red
balls at 2000 balls: 0.3345, Difference: 0.0155
Simulation 96: Proportion of red balls at 1000 balls: 0.7140, Proportion of red
balls at 2000 balls: 0.7060, Difference: 0.0080
Simulation 97: Proportion of red balls at 1000 balls: 0.3580, Proportion of red
balls at 2000 balls: 0.3585, Difference: 0.0005
Simulation 98: Proportion of red balls at 1000 balls: 0.7520, Proportion of red
balls at 2000 balls: 0.7420, Difference: 0.0100
Simulation 99: Proportion of red balls at 1000 balls: 0.6520, Proportion of red
balls at 2000 balls: 0.6620, Difference: 0.0100
Simulation 100: Proportion of red balls at 1000 balls: 0.0410, Proportion of red
balls at 2000 balls: 0.0355, Difference: 0.0055
    \end{Verbatim}


    % Add a bibliography block to the postdoc
    
    
    
\end{document}
