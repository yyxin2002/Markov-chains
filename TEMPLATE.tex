
\documentclass[openany, amssymb, psamsfonts]{amsart}
\usepackage{mathrsfs,comment}
\usepackage[usenames,dvipsnames]{color}
\usepackage[normalem]{ulem}
\usepackage{url}
\usepackage[all,arc,2cell]{xy}
\UseAllTwocells
\usepackage{enumerate}
%%% hyperref stuff is taken from AGT style file
\usepackage{hyperref}  
\hypersetup{%
  bookmarksnumbered=true,%
  bookmarks=true,%
  colorlinks=true,%
  linkcolor=blue,%
  citecolor=blue,%
  filecolor=blue,%
  menucolor=blue,%
  pagecolor=blue,%
  urlcolor=blue,%
  pdfnewwindow=true,%
  pdfstartview=FitBH}   
  
\let\fullref\autoref
%
%  \autoref is very crude.  It uses counters to distinguish environments
%  so that if say {lemma} uses the {theorem} counter, then autrorefs
%  which should come out Lemma X.Y in fact come out Theorem X.Y.  To
%  correct this give each its own counter eg:
%                 \newtheorem{theorem}{Theorem}[section]
%                 \newtheorem{lemma}{Lemma}[section]
%  and then equate the counters by commands like:
%                 \makeatletter
%                   \let\c@lemma\c@theorem
%                  \makeatother
%
%  To work correctly the environment name must have a corrresponding 
%  \XXXautorefname defined.  The following command does the job:
%
\def\makeautorefname#1#2{\expandafter\def\csname#1autorefname\endcsname{#2}}
%
%  Some standard autorefnames.  If the environment name for an autoref 
%  you need is not listed below, add a similar line to your TeX file:
%  
%\makeautorefname{equation}{Equation}%
\def\equationautorefname~#1\null{(#1)\null}
\makeautorefname{footnote}{footnote}%
\makeautorefname{item}{item}%
\makeautorefname{figure}{Figure}%
\makeautorefname{table}{Table}%
\makeautorefname{part}{Part}%
\makeautorefname{appendix}{Appendix}%
\makeautorefname{chapter}{Chapter}%
\makeautorefname{section}{Section}%
\makeautorefname{subsection}{Section}%
\makeautorefname{subsubsection}{Section}%
\makeautorefname{theorem}{Theorem}%
\makeautorefname{thm}{Theorem}%
\makeautorefname{cor}{Corollary}%
\makeautorefname{lem}{Lemma}%
\makeautorefname{prop}{Proposition}%
\makeautorefname{pro}{Property}
\makeautorefname{conj}{Conjecture}%
\makeautorefname{defn}{Definition}%
\makeautorefname{notn}{Notation}
\makeautorefname{notns}{Notations}
\makeautorefname{rem}{Remark}%
\makeautorefname{quest}{Question}%
\makeautorefname{exmp}{Example}%
\makeautorefname{ax}{Axiom}%
\makeautorefname{claim}{Claim}%
\makeautorefname{ass}{Assumption}%
\makeautorefname{asss}{Assumptions}%
\makeautorefname{con}{Construction}%
\makeautorefname{prob}{Problem}%
\makeautorefname{warn}{Warning}%
\makeautorefname{obs}{Observation}%
\makeautorefname{conv}{Convention}%


%
%                  *** End of hyperref stuff ***

%theoremstyle{plain} --- default
\newtheorem{thm}{Theorem}[section]
\newtheorem{cor}{Corollary}[section]
\newtheorem{prop}{Proposition}[section]
\newtheorem{lem}{Lemma}[section]
\newtheorem{prob}{Problem}[section]
\newtheorem{conj}{Conjecture}[section]
%\newtheorem{ass}{Assumption}[section]
%\newtheorem{asses}{Assumptions}[section]

\theoremstyle{definition}
\newtheorem{defn}{Definition}[section]
\newtheorem{ass}{Assumption}[section]
\newtheorem{asss}{Assumptions}[section]
\newtheorem{ax}{Axiom}[section]
\newtheorem{con}{Construction}[section]
\newtheorem{exmp}{Example}[section]
\newtheorem{notn}{Notation}[section]
\newtheorem{notns}{Notations}[section]
\newtheorem{pro}{Property}[section]
\newtheorem{quest}{Question}[section]
\newtheorem{rem}{Remark}[section]
\newtheorem{warn}{Warning}[section]
\newtheorem{sch}{Scholium}[section]
\newtheorem{obs}{Observation}[section]
\newtheorem{conv}{Convention}[section]

%%%% hack to get fullref working correctly
\makeatletter
\let\c@obs=\c@thm
\let\c@cor=\c@thm
\let\c@prop=\c@thm
\let\c@lem=\c@thm
\let\c@prob=\c@thm
\let\c@con=\c@thm
\let\c@conj=\c@thm
\let\c@defn=\c@thm
\let\c@notn=\c@thm
\let\c@notns=\c@thm
\let\c@exmp=\c@thm
\let\c@ax=\c@thm
\let\c@pro=\c@thm
\let\c@ass=\c@thm
\let\c@warn=\c@thm
\let\c@rem=\c@thm
\let\c@sch=\c@thm
\let\c@equation\c@thm
\numberwithin{equation}{section}
\makeatother

\bibliographystyle{plain}

%--------Meta Data: Fill in your info------
\title{Template for REU papers}

\author{Your Name}

\date{DEADLINES: Draft AUGUST 14 and Final version AUGUST 28, 2021}

\begin{document}

\begin{abstract}

This is a sample latex document with emphasis on using math mode and
equation environments.  You should use it as a template for your paper.
Some pointers are included. 
Remember that a first draft must be submitted to mentors by August 14.  The completed paper must be submitted by August 28, unless  permission for a later date has been obtained from the director of the program.   Please  name your file  Lastname.pdf.  If you have a common last name, for example Li or Wu, please name your file Lastname,Firstname.pdf.  That saves me the trouble of renaming.


\end{abstract}

\maketitle

\tableofcontents

\section{Writing}  Before getting to latex comments, I'll say a few words about writing, venting from many years of hard experience. 

It is important that what you write is something you would actually like to read.  There should be no overuse of symbols.
Clusters of symbols can be barbaric.  There are subjects whose literature is festooned with sentences written almost entirely with symbols like $\exists$, $\forall$.  Don't do that. Never use a symbol for a verb.  Never start a sentence with a symbol or with the word ``And''.

Grammar should be correct.  Never start a sentence with ``Where ...".   
Mismatches of singular and plural are excruciatingly painful, utterly abhorrent.  You cannot write
``Let ..., then ..."  That is what is called a run-on sentence. You must write ``Let ... .  Then ... ."
Alternatively, ``If ..., then ..."  works just fine.

Avoid words or phrases like Simply, Obviously, Just, ..., It is easy to see.  They serve only to intimidate or to browbeat the reader into acquiescence. 
Especially if English is not your native language, make sure you have  somebody fluent in English check what you have written.     

References to numbered statements are best in the form Theorem 2.3, not just 2.3.  Use  \verb|usepackage{hyperref}| to get this right.  Really.  I will likely return papers unread that don't. 
Equations should be referred to as (2.3), NOT 2.3 or equation 2.3 or equation (2.3).


Bad writing makes for unpleasant papers, no matter how good the material.


\section{What does the table of contents command do?}

The table of contents command will automatically make a contents.
You must run tex at least twice for this to work.  Do use \verb|\maketitle| and \verb|\tableofcontents}|.  I will likely return papers unread that don't. 

\section{How do the environment commands work?}

``Environments'' are commands that are given using the \verb|\begin{}|
  and \verb|\end{}| syntax. In the preamble, you can see we've defined
  a bunch of theorem-type environments.  For example, to get a definition, 
you type:

\begin{defn}  This is how to define a definition.
\end{defn}

And for a theorem and its proof you type:

\begin{thm}
This is the statement of a theorem.
\end{thm}
\begin{proof}
And this shows that the statement is correct.
\end{proof}

Note that the numbering is taken care of automatically, and that we've predefined a bunch of these sorts of environments in the header.  These give environments for  lemmas, corollaries and such like.  All statements are numbered in a single sequence.  An old-fashioned number system counts theorems in one sequence, definitions in another, and so forth.  Do not use that unhelpful system.  Numbers are there to help people find things.  

Another useful kind of enviroment is the equation environment.  Equations
get numbered in sequence with statements, as for example

\begin{equation}  e = mc^2
\end{equation}

Note that if you do not want a numbered equation, you can use the
environment ``equation*''
 like so:

\begin{equation*}
e=mc^2
\end{equation*}
But a quicker way to tex the same command and get the same displayed result is:
\[  e=mc^2 \]

To avoid unwanted indentations after a display, either do not skip a line or use the command \noindent.

There are plenty of other equation-type enviroments that allow you to
align several equations and such like things. The AMS's guide \cite{amsshort} is a
good place to start with these.

You can also typeset math directly in a paragraph by placing it within
dollar signs.  This is called ``math mode.''  For example: Let $e$
be energy, $m$ be momentum and $c$ be the speed of light.  Then
Einstein's famous equation says that $e=mc^2$.  This is useful, but
remember that it is harder to read inline math than displayed math. 

Remember that letters get put in a different font in math mode, so
whenever you are referencing a mathematical object you should always
put it in dollar signs.  For example, $f$ is a function, but f is just
a random letter.

Both \cite{notsoshort} and \cite{amsshort} have good lists of other symbols you can use in math mode.
These include greek letters ($\alpha, \beta, \Gamma, \Delta$),
operators ($\otimes, +, \sum$) and much more ($\leq, \diamond$).

\section{Xypic and diagrams}

If you want to draw diagrams, you should use xypic or the like.  It's actually
much easier than it looks, and we've already included it in the header
above.  Here is an example. 
\[\xymatrix{
FX \ar[r]^-{Ff} \ar[d]_{\eta_X} & FY \ar[d]^{\eta_Y} \\
GX \ar[r]_-{Gf} & GY\\} \]

\section*{Acknowledgments}  You should thank anyone who deserves thanks, and for sure you should
thank your mentor.   ``It is a pleasure to thank my mentor, 
his/her name, for ....  ".   Or add anyone else, for example ``I thank [another participant] for helping 
me understand [something or other]"

\section{bibliography}  The bibliography should list all sources that you have used and referenced.
And you should reference anything you use.   Especially if you quote any result without proof, you MUST
give a reference.   And never ever should you copy material directly or more or less directly, from a source.
If you include a figure or diagram from some source, that source should be identified.

\begin{thebibliography}{9}

\bibitem{ams} http://www.ams.org/publications/authors/tex/amslatex

\bibitem{amsshort}
Michael Downes.
Short Math Guide for \LaTeX.
http://tex.loria.fr/general/downes-short-math-guide.pdf

\bibitem{May}
J. P. May.
A Concise Course in Algebraic Topology.
University of Chicago Press. 1999. 

\bibitem{notsoshort}
Tobias Oekiter, Hubert Partl, Irene Hyna and Elisabeth Schlegl.
The Not So Short Introduction to \LaTeX 2e.
https://tobi.oetiker.ch/lshort/lshort.pdf

\end{thebibliography}

\end{document}

